\section{Lecture 1: Probability Models and Axioms}
    \subsection{Reading}
    \begin{itemize}
        \item Sections 1.1, 1.2
    \end{itemize}

    \subsection{Lecture outline}
    \begin{itemize}
        \item Probability as a mathematical framework for: 
        \begin{itemize}
            \item reasoning about uncertainty
            \item developing approaches to inference problems
        \end{itemize}

        \item Probabilistic models
        \begin{itemize}
            \item sample space
            \item probability law
        \end{itemize}

        \item Axioms of probability
        \item Simple examples
    \end{itemize}

    %% \texorpdfstring{$\Omega$}{} 这个是为了在标题中打印数学符号
    \subsection{Sample space \texorpdfstring{$\Omega$}{}}
    \begin{itemize}
        \item "List" (set) of possible outcomes
        \item List must be:
        \begin{itemize}
            \item Mutually exclusive(互斥)
            \item Collectively exhaustive(互补)
        \end{itemize}

        \item Art: to be at the "right" granularity(粒度)
    \end{itemize}

    \subsection{Sample space: Discrete example}
    \begin{itemize}
        \item Two rolls of a tetrahedral(四面體) die
        \begin{itemize}
            \item Sample space vs. sequential description
        \end{itemize}

        \item See Figure~\ref{fig:1-1}
		 
		 \begin{figure}[h!] % h for here.
		\centering
		 \includegraphics[scale=0.7]{images/1-1}
		\caption{Sample space: Discrete example}
		 \label{fig:1-1}
		 \end{figure}
    \end{itemize}

    \subsection{Sample space: Continuous example}
    \begin{itemize}
        \item $\omega = \{(x,y) | 0 \le x,y \le 1\}$
        \item See Figure~\ref{fig:1-2}
		 
        \begin{figure}[h!] % h for here.
       \centering
        \includegraphics[scale=0.7]{images/1-2}
       \caption{Sample space: Continuous example}
        \label{fig:1-2}
        \end{figure}
    \end{itemize}

    \subsection{Probability axioms(公理)}
    \begin{itemize}
        \item \textbf{Event:} a subset of the sample space
        \item See Figure~\ref{fig:1-3}
		 
        \begin{figure}[h!] % h for here.
       \centering
        \includegraphics[scale=0.7]{images/1-3}
       \caption{Event: a subset of the sample space}
        \label{fig:1-3}
        \end{figure}

        \item Probability is assigned to events
        
        \rule{\textwidth}{1pt} %横线
        \item \textbf{Axioms:}
        \item 1. \textbf{Nonnegativity:} $P(A) \geq 0$
        \item 2. \textbf{Normalization:} $P(\Omega) = 1$
        \item 3. \textbf{Additivity:} If $A \cap B = \emptyset$, then $P(A \cup B) = P(A)+ P(B)$
        \rule{\textwidth}{1pt} %横线

        \item See Figure~\ref{fig:1-4}
		 
        \begin{figure}[h!] % h for here.
       \centering
        \includegraphics[scale=0.7]{images/1-4}
       \caption{Axioms: Additivity}
        \label{fig:1-4}
        \end{figure}
            
        \begin{eqnarray*}
            1 & \overset{(2)}{=} & P(\Omega) = P(A \cup A^c) \\
            & \overset{(3)}{=} & P(A) + P(A^c) \\
            P(A) & = & 1 - P(A^c) \overset{(1)}{\leq} 1
        \end{eqnarray*}

        \item See Figure~\ref{fig:1-5}
        \begin{figure}[h!]
            \centering
            \includegraphics[scale=0.7]{images/1-5}
            \caption{A and complement of A}
            \label{fig:1-5}
        \end{figure}

        \begin{eqnarray*}
            && P(A \cup B \cup C) \\
            && = P((A \cup B) \cup C) \\
            && = P(A \cup B) + P(C) \\
            && = P(A) + P(B) + P(C)
        \end{eqnarray*}

        If $A_1,A_2,...,A_n$ disjoint.
        the $P(A_1 \cup ... \cup A_n) = P(A_1)+...+P(A_n)$

        \item See Figure~\ref{fig:1-6}
        \begin{figure}[h!]
            \centering
            \includegraphics[scale=0.7]{images/1-6}
            \caption{A,B,C in an $Omega$}
            \label{fig:1-6}
        \end{figure}

        \begin{eqnarray*}
            P(\{s_1, s_2, \dots, s_k\}) & = & P(\{s_1\}) + \dots + P(\{s_k\}) \\
            & = & P(s_1) + \dots + P(s_k)
        \end{eqnarray*}

        \item See Figure~\ref{fig:1-7}
        \begin{figure}[h!]
            \centering
            \includegraphics[scale=0.7]{images/1-7}
            \caption{finite elements in an $Omega$}
            \label{fig:1-7}
        \end{figure}


        \item Axiom 3 needs strengthening(加强)
        \item Do weird sets have probabilities?
        
    \end{itemize}

\subsection{Probability law: Example with finite sample space}
    \begin{itemize}
        \item See Figure~\ref{fig:1-8}
        \begin{figure}[h!]
            \centering
            \includegraphics[scale=0.7]{images/1-8}
            \caption{Probability law: Example with finite sample space}
            \label{fig:1-8}
        \end{figure}
        
        \item Let every possible outcome have probability 1/16
        \begin{itemize}
            \item $ P((X,Y) \; is \; (1,1) \; or \; (1,2)) = 2/16$
            \item $ P({X=1}) =  4/16$
            \item $ P(X+Y \; is \; odd) = 8/16$
            \item $ P(min(X,Y)=2) = 5/16$
        \end{itemize}
    \end{itemize}

\subsection{Discrete uniform law}
    \begin{itemize}
        \item Let all outcomes be equally likely
        \item Then, 
        $$
        P(A) = \frac{number \; of \; elements \; of \; A}{total  \; number \; of \; sample \; points}
        $$
        \item Computing probabilities $\equiv$ counting
        \item Defines fair coins, fair dice, well-shuffled card decks
    \end{itemize}

\subsection{Continuous uniform law}
    \begin{itemize}
        \item Two "random" numbers is [0,1].
        \item See Figure~\ref{fig:1-9}
        \begin{figure}[h!]
            \centering
            \includegraphics[scale=0.7]{images/1-9}
            \caption{Continuous uniform law}
            \label{fig:1-9}
        \end{figure}
        \item \textbf{Uniform} law: Probability \equiv Area 
        \begin{itemize}
            \item $P(X+Y \leq 1/2) = \frac{1}{2} \cdot \frac{1}{2} \cdot \frac{1}{2} = \frac{1}{8}$
            \item $P((X,Y)=(0.5, 0.3)) = 0$
        \end{itemize}
    \end{itemize}

\subsection{Probability law: Ex. w/countably infinite sample space}
    \begin{itemize}
        \item Sample space: $\{1,2,...\}$
        \begin{itemize}
            \item We are given $P(n) = 2^{-n}, n=1,2,...$
            \item Find $P(outcome \; is \; even)$
            \item See Figure~\ref{fig:1-10}
            \begin{figure}[h!]
                \centering
                \includegraphics[scale=0.7]{images/1-10}
                \caption{Probability law: Ex. w/countably infinite sample space}
                \label{fig:1-10}
            \end{figure}
            $$
            P(\{2,4,6,...\}) = P(2) + P(4) + ... = \frac{1}{2^2} + \frac{1}{2^4} + \frac{1}{2^6}
            $$
        \end{itemize}

        \item Countable Additivity axiom (needed for this calculation)(Santiago comment: It is \textbf{infinite} on this example instead of finite on before one ): \\
        If $A_1, A_2, ...$ are disjoint events, then:
        $$
        P(A_1 \cup A_2 \cup ...) = P(A_1) + P(A_2) + ...
        $$
        
    \end{itemize}


